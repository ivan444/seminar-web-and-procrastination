\documentclass[11pt,twocolumn,english]{article}
\usepackage[utf8]{inputenc}
\usepackage[croatian]{babel}
\usepackage[T1]{fontenc}
\usepackage{graphicx}
\usepackage{amsmath}
\usepackage{lmodern}
%\usepackage{times}
\usepackage{titlesec}
\usepackage[affil-it]{authblk}

\pagestyle{empty}

\setlength{\topmargin}{-35pt}
\setlength{\textheight}{24cm}
\setlength{\textwidth}{16cm}
\setlength{\columnsep}{0.6cm}
\setlength{\oddsidemargin}{3pt}
\setlength{\evensidemargin}{3pt}
\setlength{\parindent}{0.5cm}

\let\LaTeXtitle\title
\renewcommand{\title}[1]{\LaTeXtitle{\Large \textbf{#1}}}
\renewenvironment{abstract}
{\noindent \large \bf Sažetak.\normalsize\begin{it}}
{\end{it}\\}

\titleformat{\section}{\large\bfseries}{\thesection.}{1em}{}
\titleformat{\subsection}{\large\bfseries}{\thesubsection.}{1em}{}
\titleformat{\subsubsection}{\large\bfseries}{\thesubsubsection.}{1em}{}

\newenvironment{keywords}
{\noindent {\large {\bf Ključne riječi}}.~}{}

% Odvajanje autora, authblk paket
\renewcommand\Authsep{ \quad }
\renewcommand\Authand{ \quad }
\renewcommand\Authands{ \quad }

\begin{document}

\title{Web and Procrastination}
\author{Vedrana Janković}
\author{Zoran Hranj}
\author{Ivan Krišto}
\affil{Fakultet elektrotehnike i računarstva\\Unska 3, 10000 Zagreb,
Hrvatska}
\affil{\{vedrana.jankovic, zoran.hranj, ivan.kristo\}@fer.hr}

% Uklanjanje datuma
\date{}

\maketitle
\thispagestyle{empty}
\pagestyle{empty}

\begin{abstract}
Kad vam se jede nešto slano ili slatko, ma jednostavno kad vam se nešto jede,
palačinke su obično najbolji izbor jer sve potrebne sastojke već imate kod kuće.
Malo jaja, mlijeka, vode i brašna naći će se u svakoj kuhinji i sve što je
potrebno da dođete do slasnog zalogaja je malo dobre volje!\footnote{Sve izvukao sa: http://www.crochef.com/index .php?page=article\&id=17}
\end{abstract}

\begin{keywords}
DARPK, FER, Palačinke.
\end{keywords}

\section{Definirajmo to nešto tanko slatko}
Da Vam netko kaže da su palačinke  “vrsta tankog tijesta pripremljena od slatke
smjese koja se peče na tavi“ vjerojatno biste pomislili da uzalud komplicira – pa
svi znaju što su palačinke! I bili biste u pravu, gotovo svaka kultura ima u svom
gastronomskom repertoaru barem jednu vrstu palačinki. Ono što im je svima
zajedničko je da se pripremaju od osnovinih namirnica kao što su jaja, brašno i
voda ili mlijeko te da se prže na maloj količini masnoće. Uglavnom se pripremaju
bez dodataka za dizanje tijesta, iako se u neke vrste dodaje prašak za pecivo ili
soda bikarbona. Jako su rijetki recepti u kojima u palačinke ide kvasac.

\subsection{Mali trikovi za palačinke kakve svi vole}
Mnogi vole da su palačinke mekane, tanke, lagano slatkaste i da nisu previše
masne. Mekoća palačinki ovisi o vrsti tekućine koja se koristi za izradu smjese.
Palačinke u kojima je dio mlijeka zamijenjen vodom bit će nešto laganije od onih
u kojima se koristi samo mlijeko. Želite li pak dodatno pojačati taj efekt možete
koristiti gaziranu mineralnu vodu dok neki recepti čak traže uporabu piva.

Debljina palačinki naravno opet ovisi o vrsti smjese – palačinke s
praškom za pecivo kao što su američke “pancakes“ bit će debele centimetar do centimetar i
pol dok će palačinke rasprostranjene u našim krajevima biti mnogo “elegantnije“.
Za tanke palačinke važno je koristiti tavu s neprijanjajućom podlogom
(napravljenom od teflona) na koju se po želji može kapnuti koja kap ulja da bi se
posve isključila mogućnost lijepljenja. Zatim se manju količninu tekuće smjese
kružnim pokretima rasporedi po cijeloj površini tave tako da bude što
ravnomjernija, nožem se prođe između tave i palačinke i nakon minute do dvije – u
ovisnosti o jačini vatre – se palačinka okrene. Nešto spretniji mogu izvoditi
vratolomije bacajući palačinke u zrak, a oni koji pak vole sigurnije metode lako
palačinku mogu prevrnuti služeći se nešto duljim nožem.

Za ljubitelje slatkastih palačinki, idealno je rješenje dodavanje kesice šećer
vanilije u osnovnu smjesu. Dat će palačinkama finu aromu, malo će ih zasladiti, a
opet neće biti pretjerano slatke. Prevelika količina šećera u tijestu za
palačinke čini ga ljepljivijim pa je samim tim i teže rukovati palačinkama
prilikom prženja. Osim vanilije, nezaobilazna aroma za pržena tijesta je limunova
korica. Naribana korica pola limuna i najobičnije će palačinke učiniti posebnima.

Što se pak masnoće tiče, potrebno je obratiti pozornost na dvije stvari. Prva je,
naravno, količina masnoće koja se koristi bilo u samoj smjesi za palačinke, bilo
za prženje. Da bi se za prženje koristilo što manje masnoće, praktična je uporaba
silikonskih kistova koji su otporni na visoke temperature. Pomoću takve vrste
kista i najmanja količina msanoće se jednostavno rasporedi po čitavoj tavi. U
slučaju da pak nemate silikonski kist, možete jednostavno lagano namastiti jedan
papirnati kuhinjski ručnik i njim premazati dno tave.

I posljednje, ali ne i najmanje važno: masnoću u prženim tijestima jako dobro
neutralizira manja količina alkohola. Za palačinke je najpraktičnije koristiti
malo ruma koji ima dvije prednosti: masnoća se manje osjeća, a njegova aroma se
provlači kroz palačinku poput glazbenog leitmotiva.

\subsubsection{Što se sve s palačinkama može}
Prvo što pada na pamet – svašta! Krenimo od najjednostavnijih varijanti: možete
ih posipati šećerom i mljevenim orasima, dodati im preljev od čokolade, karamela
ili na bazi voća, možete ih premazivati najrazličitijim kombinacijama: džemom ili
marmeladom, krem namazima od čokolade ili lješnjaka ili pak medom. Puniti ih
možete svježim voćem ili sladoledom. Američke “pancakes“ učinit ćete još
autentičnijima poslužite li ih naslagane na hrpu i prelivene javorovim sokom.

Međutim, palačinke su izvanredna podloga i za mnogo složenije kombinacije okusa.
Npr. palačinke punjene svježim kravljim sirom mogu se pripremiti i u slatkoj i u
slanoj varijanti. Ono što ovo jelo čini neodoljivima je faza br. 2: ovako
pripremljene palačinke potrebno je malo zapeći u pećnici. Naročito su zanimljivi
recepti za tzv. torte od palačinki u kojima se palačinek slažu jedna na drugu, a
između njih se stavlja nadjev pa se u većini slučajeva tako složena “torta“
stavlja u pećnicu.

Što se slanih palačinki tiče, njih osim kravljim sirom možete puniti različitim
vrtsma gljiva kao što su šampinjoni i vrganji pa preliti umakom bešamel i također
zapeći. Idemo dalje:  palačinke punjene samim špinatom ili kombinacijom špinata i
sira izvući će osmijeh na ljubitelje slanoga. Osim vegetarijanskih opcija, na
izboru Vam stoje i mesna punjenja: sjeckana piletina ili neka druga vrsta mesa
učinit će palačinke glavnom zvijezdom obroka.

Osim što se možete igrati različitim vrstama punjenja, imate i mogućnost
variranja sastojaka za osnovnu smjesu palačinki. Najočitiju razliku u smjesi čini
vrsta brašna koju koristite. Odlučite li se upustiti u avanturu američkih
palačinki, preporučamo Vam korištenje brašna za dizana tijesta. Slana se pak
punjenja jako dobro slažu s palačinkama od heljdinog brašna koje su specijalitet
francuske pokrajine Bretanje. U tom dijelu Francuske nazivaju se “galettes“ i
jedno su od jela kojima se taj dio zemlje sira i vina jako ponosi. U Kini su
raširene palačinke napravljene od rižinog brašna – one su nešto svjetlije, a
često se pune svinjetinom i serviraju kao glavno jelo.

U nešto širem smislu riječi, palačinkama se mogu smatrati čak i tortilje. Prave
se od kukuruznog brašna, a u meksičkoj kuhinji su jednostavno nezaobilazne.
Poslužuju se i tople i hladne, i kao predjelo, i kao glavno jelo, a ponekad čak i
kao dio deserta. Meksičke kuharice pune ih mljevenim mesom, grahom, sirom,
povrćem i najrazličitijim kombinacijama svega nabrojanog. A za desert nakon
tortilje punjene mesom mogli biste dobiti opet – palačinke! Ovaj put radilo bi se
o vrsti pripremljenoj s nešto kakaa u samoj smjesi što bi im dalo karakterističnu
tamnu boju.

Upravo je ova gotovo neograničena mogućnost kombiniranja i primjene ono što je
palačinke učinilo omiljenima u svim krajevima svijeta pa stoga i ne čudi golemi
broj recepata za nešto što naoko djeluje tako jednostavno. Sigurni smo da i Vi
imate svog asa u rukavu...

\nocite{Banko01mitigatingthe}
\nocite{ivankovic2009lcs}
\nocite{CoCoTehnickaDok}
\nocite{DBLP:series/sci/2007-43}
\nocite{Finn01factor}
\nocite{panian05informatickirjecnik}

\bibliography{literatura}
\bibliographystyle{plain}

\end{document}
