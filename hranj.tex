\section{Razlozi prokrastinacije}
\TODO{malo uvoda o razlozima}
Psihoneuroza (engl. \emph{neuroticism}), jest stanje vrlo slično brizi,
anksioznosti i odbojnosti. Iracionalna vjerovanja i razmišljanja o nekoj temi
mogu ju učiniti nezanimljivom i neprivlačnom, pobuditi anksioznost što kod
prokrastinatora izaziva izbjegavanja teme i traženja alternativnih aktivnosti.
Temeljna iracionalna vjerovanja koja vode prokrastinaciji su vjerovanje da je
osoba nedorasla i vjerovanje da je svijet presložen i prezahtjevan. Usko
povezane su pojave niskog samopoštovanja i nastojanja osobe da sama sebi oteža
obavljanje bilo kakve radnje (engl. self-handicapping). Istraživanja
\cite{Caballero95neuro,Saklofske95neuro} su pokazala da psihoneuroza uvelike
povečava prijemljivost prema depresiji, uz koju se često povezuje i pesimizam i
osjećaj nemoći.

\section{Posljedice prokrastinacije}
Činjenica da je osoba neuspješna u organizaciji vlastitog vremena povezana je
općenito sa padom raspoloženja i efikasnosti. Prokrastinacija inicijalno
poboljšava raspoloženje (odlaganje neželjenog posla za kasnije), ali u
konačnici ga pogoršava. Primjećujemo da loše raspoloženje ne samo da može biti
uzrok prokrastinacije, već i njezina posljedica. Jednako kao i s raspoloženjem,
efikasnost dozvoljava mogućnosti recipročne veze, tj. prokrastinacija može
dovesti do pada efikasnosti, uzrokujući smanjenje samopoštovanja, što opet
rezultira prokrastinacijom.

\section{Suočavanje s prokrastinacijom}
Prvi korak u suočavanju s prokrastinacijom (kao i sa svakim drugim problemom)
je njezino primjećivanje tj. prihvaćanje njezinog postojanja. 

U većini slučajeva, najmudriji način nošenja s prokrastinacijom je obavljanje
neugodnih zadaća što ranije moguće, dok još ima vremena da se obavi kvalitetno.

"Disciplina je... 1. Učini što se mora učiniti; 2. Kad se mora učiniti; 3. Sve
dok se može učiniti; 4. Postupi tako svaki put." - Bobby Knight  

Za većinu prokrastinatora, jednostavna lista zadataka (engl. \emph{to-do-list})
bit će dovoljna \cite{TuckerPsySelfHelp}. Za malo tvrdokornije, postoji metoda
poznata pod nazivom "petminutni plan". Ideja je podijeliti zadaću na manje
dijelove te se obvezati na rad samo pet minuta dnevno. Mnogi ljudi otkriju kako
im nije problem utrošiti i više od pet minuta na rad. Ključ je steći naviku
počinjanja rješavanja zadaća što ranije. Svakodnevnom vježbom, rano počinjanje
može prijeći u naviku čime se problem prokrastinacije uvelike umanjuje. Iako
služi kao velika pomoć u učenju, Internet može biti i izvor ometanja prilikom
učenja. Stranica \cite{ColDeg25ways} predstavlja 25 preporuka koje mogu pomoći
pri smanjivanju negativnog utjecaja Interneta prilikom učenja i obavljanja
zadataka općenito. Neke od preporuka su: isključivanje programa za trenutnu
komunikaciju preko Interneta {ima neki ljepši prijevod?}, slušanje glazbe (ako
Vam ne smeta), blokiranje najposjećenijih stranica tokom rada, definiranje
pauzi za surfanje Internetom, itd. Ako ništa drugo ne djeluje, preporuča se
isključivanje internetske veze tokom obavljaja posla.
