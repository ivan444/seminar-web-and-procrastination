\documentclass[11pt,twocolumn,english]{article}
\usepackage[utf8]{inputenc}
\usepackage[croatian]{babel}
\usepackage[T1]{fontenc}
\usepackage{graphicx}
\usepackage{amsmath}
\usepackage{lmodern}
\usepackage{titlesec}
\usepackage[affil-it]{authblk}

\pagestyle{empty}

\setlength{\topmargin}{-35pt}
\setlength{\textheight}{24cm}
\setlength{\textwidth}{16cm}
\setlength{\columnsep}{0.6cm}
\setlength{\oddsidemargin}{3pt}
\setlength{\evensidemargin}{3pt}
\setlength{\parindent}{0.5cm}

\let\LaTeXtitle\title
\renewcommand{\title}[1]{\LaTeXtitle{\Large \textbf{#1}}}
\renewenvironment{abstract}
{\noindent \large \bf Sažetak.\normalsize\begin{it}}
{\end{it}\\}

\titleformat{\section}{\large\bfseries}{\thesection.}{1em}{}
\titleformat{\subsection}{\large\bfseries}{\thesubsection.}{1em}{}
\titleformat{\subsubsection}{\large\bfseries}{\thesubsubsection.}{1em}{}

\newenvironment{keywords}
{\noindent {\large {\bf Ključne riječi}}.~}{}

% Odvajanje autora, authblk paket
\renewcommand\Authsep{ \quad }
\renewcommand\Authand{ \quad }
\renewcommand\Authands{ \quad }

\newcommand{\engl}[1]{(engl.~\emph{#1})}

\begin{document}

\title{Prokrastinacija i Internet}
\author{Vedrana Janković}
\author{Zoran Hranj}
\author{Ivan Krišto}
\affil{Fakultet elektrotehnike i računarstva\\Unska 3, 10000 Zagreb,
Hrvatska}
\affil{\{vedrana.jankovic, zoran.hranj, ivan.kristo\}@fer.hr}

% Uklanjanje datuma
\date{}

\maketitle
\thispagestyle{empty}
\pagestyle{empty}

\begin{abstract}
Napisati\ldots
\end{abstract}

\begin{keywords}
Prokrastinacija, Internet
\end{keywords}

\section{Uvod}
Danas je uobičajeno u radnom prostoru imati pristup Internetu. Uvođenjem
širokopojasnog \engl{broadband} pristupa, veza s Internetom je postala stalna,
brža i dostupnija.

Količina, raznolikost i multimedijalnost sadržaja Internet čini izuzetno
pogodnim za prokrastinaciju.

Iznosimo pregled korištenja Interneta u svrhu prokrastiniranja radi razvoja
metode eliminacije Interneta kao izvora za prokrastiniranje. Opisati ćemo
utjecaj Interneta na razvoj prokrastinacije, no ne i njegov potencijal za
sprječavanje prokrastiniranja.

U 2.~odjeljku dana je neformalna definicija prokrastinacije te komentirana
prokrastinacija korištenjem Interneta u odnosu na ovisnost o Internetu. Slijedi
3.~poglavlje u kojem je navedeno zašto ljudi prokrastiniraju, te 4.~u kojem su
opisane posljedice prokrastinacije. Način korištenja Interneta za
prokrastiniranje uz primjere za dio razloga prokrastiniranja nevedeni su u
5.~odjeljku. Zaključak je dan u 6.~odjeljku.


\section{Definicija prokrastinacije}
Prokrastinacija je čin nepotrebnog odgađanja obaveza koji dovodi do 
doživljavanja subjektivne nelagode \cite{solomon1984academic}. Razlika između
prokrastinacije i jednostavnog odlaganja poslova je upravo u tom osjećaju
nelagode (``\emph{grizodušja}'') radi posla koji nije obavljen
\cite{burka2004procrastination}.

Često se svjesno odlučimo prokrastinirati. Odlažemo posao radi njegovog niskog
prioriteta ili potrebe da primislimo o dijelovima prije donošenja konačne odlike
ili samog izvršavanja. Prokrastinaciju koristimo da si damo vremena za
razrješavanje mogućnosti ili fokusiranje na stvari koje nam se čine bitnijima
\cite{burka2004procrastination}.

Sindrom preokupacije Internetom se može opisati kao provođenje pretjerane
količine vremena na Internetu, poteškoća u upravljanju na Internetu provedenim
vremenom, osjećaj da je ``vanjski svijet'' dosadan, razdražljivost, depresiju
te česte promjene raspoloženja kad je pristup Internetu ograničen ili
razdražljivost na ometanje dok koristi na Internet
\cite{yellowlees2007problematic}.
 
Za razliku od ovisnosti o Internetu, osoba koja koristi Internet za
prokrastinaciju nije razdražljiva dok ga koristi ili ako joj je pristup
Internetu ograničen te zbog prokrastinacije na Internetu osoba ``vanjski
svijet'' neće smatrati dosadnim.


%\section{Razlozi prokrastriniranja i potrebe prokrastinatora}
\section{Razlozi prokrastiniranja}
% TODO: malo uvoda o razlozima

Psihoneuroza \engl{neuroticism}, jest stanje vrlo slično brizi,
anksioznosti i odbojnosti. Iracionalna vjerovanja i razmišljanja o nekoj temi
mogu ju učiniti nezanimljivom i neprivlačnom, pobuditi anksioznost što kod
prokrastinatora izaziva izbjegavanja teme i traženja alternativnih aktivnosti.
Temeljna iracionalna vjerovanja koja vode prokrastinaciji su vjerovanje da je
osoba nedorasla i vjerovanje da je svijet presložen i prezahtjevan. Usko
povezane su pojave niskog samopoštovanja i nastojanja osobe da sama sebi oteža
obavljanje bilo kakve radnje \engl{self-handicapping}. Istraživanja
\cite{Caballero95neuro, Saklofske95neuro} su pokazala da psihoneuroza uvelike
povečava prijemljivost prema depresiji, uz koju se često povezuje i pesimizam i
osjećaj nemoći.

Od konkretnijih razloga \cite{PickBrain5reasons,Lifehack6reasons} navodimo
sljedeće. Odbojni poslovi su najćešći razlog prokrastinacije. Nitko ne uživa
obavljajući poslove koji su mu neprivlačni. Dodatno, složeni poslovi ili
projekti s mnogo koraka mogu djelovati krajnje neinteresantno ili jednostavno
preteški za nas. Takvi poslovi se još nazivaju i mentalne planine (engl.
\emph{mental mountains}).

Život u moderno doba može biti umarajući. Brojne aktivnosti koje se
događaju oko pojedinaca, sitnice koje im plijene pozornost i poslovi koji ih
čekaju mogu biti iscrpljujući. Zato nije čudno da se ljudi ponekad osjećaju
potpuno klonulo i bez volje za ičim. U ovom slučaju, prokrastinacija može biti i
korisna ukoliko se neki posao odgađa u svrhu odmora.

\section{Posljedice prokrastinacije}
Činjenica da je osoba neuspješna u organizaciji vlastitog vremena povezana je
općenito sa padom raspoloženja i efikasnosti. Prokrastinacija inicijalno
poboljšava raspoloženje (odlaganje neželjenog posla za kasnije), ali u
konačnici ga pogoršava.

Stanje u koje se dovode prokrastinatori se kreće od
stanja žaljenja, pa sve do samoosude ili čak očaja. Frustrirani su i ljuti jer
smatraju da ih je prokrastinacija spriječila u obavljanju poslova za koje
smatraju da su sposobni \cite{burka2004procrastination}.

Primjećujemo da loše raspoloženje ne samo da može biti
uzrok prokrastinacije, već i njezina posljedica. Jednako kao i s raspoloženjem,
efikasnost dozvoljava mogućnosti recipročne veze, tj.~prokrastinacija može
dovesti do pada efikasnosti, uzrokujući smanjenje samopoštovanja, što opet
rezultira prokrastinacijom.


\section{Internet u službi prokrastinacije}
% TODO: Napisati


%\section{Suočavanje s prokrastiniranjem putem Interneta}
\section{Suočavanje s prokrastinacijom}
% TODO: Možda naslov staviti: ``Suočavanje s prokrastiniranjem putem
% Interneta''?

Prvi korak u suočavanju s prokrastinacijom (kao i sa svakim drugim problemom)
je njezino primjećivanje tj.~prihvaćanje njezinog postojanja. 

U većini slučajeva, najmudriji način nošenja s prokrastinacijom je obavljanje
neugodnih zadaća što ranije moguće, dok još ima vremena da se obavi kvalitetno.

``Disciplina je... 1.~Učini što se mora učiniti; 2.~Kad se mora učiniti; 3.~Sve
dok se može učiniti; 4.~Postupi tako svaki put.'' -- Bobby Knight  

Za većinu prokrastinatora, jednostavna lista zadataka \engl{to-do-list}
bit će dovoljna \cite{TuckerPsySelfHelp}. Za malo tvrdokornije, postoji metoda
poznata pod nazivom ``petminutni plan''. Ideja je podijeliti zadaću na manje
dijelove te se obvezati na rad samo pet minuta dnevno. Mnogi ljudi otkriju kako
im nije problem utrošiti i više od pet minuta na rad. Ključ je steći naviku
počinjanja rješavanja zadaća što ranije. Svakodnevnom vježbom, rano počinjanje
može prijeći u naviku čime se problem prokrastinacije uvelike umanjuje. Iako
služi kao velika pomoć u učenju, Internet može biti i izvor ometanja prilikom
učenja. Stranica \cite{ColDeg25ways} predstavlja 25 preporuka koje mogu pomoći
pri smanjivanju negativnog utjecaja Interneta prilikom učenja i obavljanja
zadataka općenito. Neke od preporuka su: isključivanje programa za trenutnu
komunikaciju preko Interneta \engl{instant messaging tools}, slušanje
glazbe (ako Vam ne smeta), blokiranje najposjećenijih stranica tokom rada, definiranje
pauzi za surfanje Internetom, itd. Ako ništa drugo ne djeluje, preporuča se
isključivanje internetske veze tokom obavljaja posla.


\section{Zaključak}
% TODO: Napisati


\bibliography{literatura}
\bibliographystyle{plain}

\end{document}
