\documentclass[11pt,english]{article}
\usepackage[utf8]{inputenc}
\usepackage[croatian]{babel}
\usepackage[T1]{fontenc}
\usepackage{graphicx}
\usepackage{amsmath}
\usepackage{lmodern}
\usepackage{titlesec}
\usepackage{url}
\usepackage[affil-it]{authblk}
\usepackage{multicol}

\pagestyle{empty}

\setlength{\topmargin}{-35pt}
\setlength{\textheight}{24cm}
\setlength{\textwidth}{16cm}
\setlength{\columnsep}{0.6cm}
\setlength{\oddsidemargin}{3pt}
\setlength{\evensidemargin}{3pt}
\setlength{\parindent}{0.5cm}

\let\LaTeXtitle\title
\renewcommand{\title}[1]{\LaTeXtitle{\Large \textbf{#1}}}

\renewenvironment{abstract}
{\noindent \large \bf Sažetak.\normalsize\begin{it}}
{\end{it}\\}

\titleformat{\section}{\large\bfseries}{\thesection.}{1em}{}
\titleformat{\subsection}{\large\bfseries}{\thesubsection.}{1em}{}
\titleformat{\subsubsection}{\large\bfseries}{\thesubsubsection.}{1em}{}

\newenvironment{keywords}
{\noindent {\large {\bf Ključne riječi}}.~}{}

% Odvajanje autora, authblk paket
\renewcommand\Authsep{ \quad }
\renewcommand\Authand{ \quad }
\renewcommand\Authands{ \quad }

\newcommand{\engl}[1]{(engl.~\emph{#1})}

\begin{document}

\title{Prokrastinacija i Internet}
\author{Vedrana Janković}
\author{Zoran Hranj}
\author{Ivan Krišto}
\affil{Fakultet elektrotehnike i računarstva\\Unska 3, 10000 Zagreb,
Hrvatska}
\affil{\{vedrana.jankovic, zoran.hranj, ivan.kristo\}@fer.hr}

% Uklanjanje datuma
\date{}

\maketitle
\thispagestyle{empty}
\pagestyle{empty}
\begin{multicols}{2}

\begin{abstract}
U ovom radu obrađena je problematika prokrastinacije. Napravljen je 
osvrt na aspekte prokrastinacije povezane s Internetom i dan je kratak 
pregled metoda koje smanjuju udio prokrastinacije tijekom rada.
\end{abstract}

\begin{keywords}
Prokrastinacija, Internet
\end{keywords}

\section{Uvod}
Danas je uobičajeno u radnom prostoru imati pristup Internetu. Uvođenjem
širokopojasnog \engl{broadband} pristupa, veza s Internetom postala je stalna,
brža i dostupnija.

Količina, raznolikost i multimedijalnost sadržaja Internet čini izuzetno
pogodnim za prokrastinaciju.

Iznosimo pregled korištenja Interneta u svrhu prokrastiniranja radi razvoja
metode eliminacije Interneta kao izvora prokrastinacije. Opisan je utjecaj 
Interneta na razvoj prokrastinacije i potencijal za sprječavanje prokrastiniranja.

U 2.~odjeljku dana je neformalna definicija prokrastinacije te komentirana
prokrastinacija korištenjem Interneta u odnosu na ovisnost o Internetu. Slijedi
3.~odjeljak u kojem je navedeno zašto ljudi prokrastiniraju te 4.~u kojem su
opisane posljedice prokrastinacije. Način korištenja Interneta za
prokrastiniranje uz primjere za dio razloga prokrastiniranja nevedeni su u
5.~odjeljku. Zaključak je dan u 6.~odjeljku.

\section{Definicija prokrastinacije}
Prokrastinacija je čin nepotrebnog odgađanja obaveza koji dovodi do 
doživljavanja subjektivne nelagode \cite{solomon1984academic}. Razlika između
prokrastinacije i jednostavnog odlaganja poslova upravo je spomenuti osjećaj
nelagode (``\emph{grizodušja}'') zbog posla koji nije obavljen
\cite{burka2004procrastination}.

Često se svjesno odlučimo prokrastinirati. Odlažemo posao zbog njegovog niskog
prioriteta ili potrebe da promislimo o dijelovima prije donošenja konačne odluke
ili samog izvršavanja. Prokrastinaciju koristimo za dobivanje vremena za
razrješavanje mogućnosti ili fokusiranje na stvari koje nam se čine bitnijima
\cite{burka2004procrastination}.

Sindrom preokupacije Internetom može se opisati kao provođenje pretjerane
količine vremena na Internetu, postojanje poteškoća u upravljanju na Internetu provedenim
vremenom, osjećaj da je ``vanjski svijet'' dosadan, razdražljivost, depresiju
te česte promjene raspoloženja kad je pristup Internetu ograničen ili
razdražljivost na ometanje tijekom korištenja Interneta
\cite{yellowlees2007problematic}.
 
Za razliku od ovisnosti o Internetu, osoba koja koristi Internet za
prokrastinaciju nije razdražljiva dok ga koristi ili ako joj je pristup
Internetu ograničen te zbog prokrastinacije na Internetu osoba ``vanjski
svijet'' neće smatrati dosadnim.


\section{Razlozi prokrastiniranja}
Pokazalo se da različite osobe prokastiniraju iz sličnih razloga
\cite{steel2007nature}. Razlozi se obično vežu uz karakteristiku zadaće, osobni
dojam o zadaći ili pristup zadaći, npr.~\cite{solomon1984academic}: strah od
vrednovanja, perfekcionizam, poteškoće u donošenju odluka, pobuna protiv
kontrole, nedostatak asertivnosti, strah od posljedica uspjeha, odbojnost
zadatka, manjak samopoštovanja te mogućnost biranja između više jednako
rangiranih poslova \cite{o2001choice}.

Dio spomenutih razloga može se opisati psihoneurozom \engl{neuroticism},
tj.~stanjem vrlo slično brizi, anksioznosti i odbojnosti. Iracionalna vjerovanja
i razmišljanja o nekoj temi mogu ju učiniti nezanimljivom i neprivlačnom te
pobuditi anksioznost što kod osobe izaziva izbjegavanja teme i traženje
alternativnih aktivnosti \cite{steel2007nature}. Temeljna iracionalna
vjerovanja koja vode prokrastinaciji su vjerovanje da je osoba nedorasla i vjerovanje da je svijet
presložen i prezahtjevan. Usko povezane su pojave niskog samopoštovanja i
nastojanja osobe da sama sebi oteža obavljanje bilo kakve radnje
\engl{self-handicapping}. Istraživanja \cite{Caballero95neuro, Saklofske95neuro}
su pokazala da psihoneuroza uvelike povećava prijemljivost prema depresiji, uz
koju se često povezuje i pesimizam i osjećaj nemoći.

U velikom broju situacija osoba mora odlučiti ne samo kada obaviti zadatak, već
i koji zadatak obaviti i koliko truda uložiti za odabrani zadatak. 
Navedeno se odlučuje po različitim kriterijima te može čak i osobu s visokom
razinom samokontrole navesti na prokrastiniranje \cite{o2001choice}.

Od konkretnijih razloga \cite{PickBrain5reasons, Lifehack6reasons} navodimo
sljedeće. Odbojni poslovi su najčešći razlog prokrastinacije. Nitko ne uživa
obavljajući poslove koji su mu neprivlačni.~Dodatno, složeni poslovi ili
projekti s mnogo koraka mogu djelovati krajnje nezanimljivima ili jednostavno
preteškima. Takvi poslovi nazivaju se još i mentalnim planinama \engl{mental
mountains}.

Život u moderno doba može biti umarajući. Brojne aktivnosti koje se
događaju oko pojedinaca, misli o poslovima koji ih čekaju i sitnice koje im
plijene pozornost mogu biti iscrpljujuće. Zato nije čudno da se ljudi ponekad
osjećaju potpuno klonulo i bez volje za ičim. U ovom slučaju,
prokrastinacija može biti i korisna ukoliko se neki posao odgađa u svrhu odmora.


\section{Posljedice prokrastinacije}
Činjenica da je osoba neuspješna u organizaciji vlastitog vremena povezana je
općenito s padom raspoloženja i efikasnosti. Prokrastinacija inicijalno
poboljšava raspoloženje (odlaganje neželjenog posla za kasnije), ali u
konačnici ga pogoršava.

Stanje u koje se dovode prokrastinatori kreće od
stanja žaljenja sve do samoosude ili čak očaja. Frustrirani su i ljuti jer
smatraju da ih je prokrastinacija spriječila u obavljanju poslova za koje
smatraju da su sposobni \cite{burka2004procrastination}.

Loše raspoloženje može biti ne samo uzrok prokrastinacije,
već i njezina posljedica. Jednako kao i s raspoloženjem, pad efikasnosti
dopušta mogućnost recipročne veze, tj.~prokrastinacija može dovesti do pada
efikasnosti, uzrokujući smanjenje samopoštovanja, što opet rezultira prokrastinacijom.


\section{Internet u službi prokrastinacije}
Karakteristična za prokrastinatore jest iracionalna percepcija vremena
fragmentiranoga u kratke intervale. Ovakvo poimanje vremena olakšava odgađanje
zadatka koji se treba obaviti. Od zadatka se ne odustaje, ali pri odabiru
aktivnosti koju će osoba obavljati u sljedećem kratkom vremenskom intervalu
pojedinac, opravdavajući se upravo kratkotrajnošću aktivnosti, odabire ``maleni
užitak'' umjesto konkretnog zadatka koji ga čeka. Ova se odluka temelji na
uvjerenju kako zadatak može pričekati tih nekoliko minuta provedenih obavljajući
ugodniju aktivnost, odnosno prokrastinirajući, i to bez zamjetnih negativnih
posljedica na obavljanje samog zadatka. Međutim, kontinuirana fragmentacija
vremena u malene isječke u kojima se pojedinac bavi ``nečim drugim'' rezultira u
stalnom odgađanju obavljanja inicijalnog zadatka \cite{Online08thatchera}.

Internet nudi neograničene mogućnosti pronalaska prethodno opisanih malenih,
kratkotrajnih i ugodnih aktivnosti čije se obavljanje može prekinuti po želji.
Internet je brz i lako dostupan te uspijeva održati privid produktivno i efikasno
utrošenog vremena. Navedene su tri pretpostavke o razlozima prokrastinacije
povezane s uporabom Interneta.

Prva pretpostavka usko je povezana sa shvaćanjem Interneta kao važnog alata pri
obavljanju raznolikih zadataka. Dostupnost velike količine informacija rezultira
učestalijim i duljim prokrastinacijskim aktivnostima, koje je lakše opravdati i
lažno objasniti njihovu korisnost, s obzirom da ih je lako percipirati kao
istraživanje \cite{Lavoie01cyberslacking}.

Sljedeća hipoteza odnosi se na Internet kao izvor opuštanja i zabavnih
informacija. Informacije se pod utjecajem televizije i Interneta sve više
trivijaliziraju, bivajući predstavljene kao kratke, iznenadne i dinamične crtice.
Znanje se na ovaj način fragmentira i dekontekstualizira, ne dopuštajući dublje
promišljanje o predstavljenim informacijama te se pretvara u zabavu po krinkom
korisnih aktivnosti.  Mnogobrojne zanimljive i nepovezane informacije do kojih na
Internetu izuzetno lako doći na ovaj način produžuju vrijeme odgađanja obavljanja
važnog zadatka \cite{Postman85amusing}. Internet, kao i televizija, može
poslužiti kao medij olakšavanja od stresa i tjeskobe. Ovo se odražava u
eskapističkim potrebama osobe pod stresom, koja bježi manje zahtjevnim i
umarajućim aktivnostima, a kakve su upravo one internetske
\cite{Lavoie01cyberslacking}.

Bitno je napomenuti i kako svijest o prokrastinaciji na Internetu dovodi do
kontraefektnih negativnih emocija i dodatne tjeskobe \cite{Kraut98internet}.

Na uzorku od 308 ljudi iz različitih dijelova Sjeverne Amerike (srednje dobi
$29.4$ godina, 198 žena) provedeno je istraživanje o navikama i načinu korištenja
Internetom. Sakupljeni podaci uključuju demografske informacije, stavove prema
Internetu, količinu vremena provedenog online (kod kuće, na poslu i u školi),
sklonost prokrastinaciji te procjenu pojave pozitivnih i negativnih emocija
prilikom korištenja Interneta. Rezultati su pokazali kako 50.7\% ispitanika
uočava učestalo prokrastiniranje prilikom korištenja Interneta te da se 47\%
vremena provedenog na Internetu svrstava u prokastinaciju. Također, utvrđeno je
kako je prokrastinacija na Internetu u pozitivnoj korelaciji s gore navedenim
hipotezama, dakle s poimanjem Interneta kao izvora zabave i gotovo neograničene
količine sadržaja, olakšanja od stresa te istovremeno, paradoksalno, važnog i
korisnog alata za rad \cite{Lavoie01cyberslacking}.

Posebno izraženo pri izvršavanju zadataka na Internetu jest prokrastiniranje zbog 
postojanja previše opcija. S obzirom na brojnost stvaratelja sadržaja, Internet
nudi veliku količinu informacija i brojnost oblika njihove prezentacije.
Potonje dovodi do neodlučnosti pri odabiru, a time i do prokrastinacije.

Primjer je traženje informacija na Internetu, npr.~želimo kupiti novo glačalo
za robu, a ne znamo na koje karakteristike moramo paziti te raspon cijena. 
Dobar izvor informacija za taj problem su internetski forumi
(npr.~\url{http://www.forum.hr}). Iz teme ``\emph{Koja pegla?}''\footnote{\url{http://www.forum.hr/showthread.php?t=129569}}
se može zaključiti da parne stanice nisu za manja kućanstva te su teške za
manipulaciju i skupe, da snaga mora biti minimalno $2000 W$, da mora
imati bar $80 g$ udarne pare te da se kvalitetne pegle (ali ne parne stanice)
nalaze u cjenovnom rangu od 300 do 700 kn. Iako je
\emph{Google}\footnote{\url{http://www.google.com}} tu stranicu vratio kao prvi
rezultat na upit ``\emph{kupovina pegle za robu}'' većina će posjetiti bar još
jednu internetsku stranicu u (nepotrebnoj) potrazi za dodatnim informacijama.


\section{Suočavanje s prokrastiniranjem putem Interneta}
Prvi korak u suočavanju s prokrastinacijom (kao i sa svakim drugim problemom)
jest njezino primjećivanje tj.~prihvaćanje njezinog postojanja unatoč
mogućoj dodatnoj tjeskobi i negativnim emocijama \cite{Kraut98internet}.

U većini slučajeva, najmudriji način nošenja s prokrastinacijom je obavljanje
neugodnih zadaća što ranije moguće, dok pojedinac još ima vremena obaviti ih kvalitetno.

``Disciplina je... 1.~Učini što se mora učiniti; 2.~Kad se mora učiniti; 3.~Sve
dok se može učiniti; 4.~Postupi tako svaki put.'' -- Bobby Knight.  

Za većinu prokrastinatora, jednostavna lista zadataka \engl{to-do-list}
bit će dovoljna \cite{TuckerPsySelfHelp}. Za malo tvrdokornije, postoji metoda
poznata pod nazivom ``petminutni plan''. Ideja je podijeliti zadaću na manje
dijelove te se obvezati na rad samo pet minuta dnevno. Mnogi ljudi otkriju kako
im nije problem utrošiti i više od pet minuta na rad. Ključ je steći naviku
počinjanja rješavanja zadaća što ranije. Svakodnevnom vježbom, rano počinjanje
može prijeći u naviku čime se problem prokrastinacije uvelike umanjuje.

Stranica \cite{ColDeg25ways} predstavlja 25 preporuka koje mogu pomoći pri
smanjivanju negativnog utjecaja Interneta prilikom učenja i obavljanja zadataka
općenito. Neke od preporuka su: gašenje programa za trenutnu komunikaciju preko
Interneta \engl{instant messaging tools}, slušanje glazbe (ukoliko osobu to ne
smeta), blokiranje najposjećenijih stranica tijekom rada, definiranje pauzi za
surfanje Internetom, itd. Postoje razni programi za pomoć pri izbjegavanju
nepotrebnog gubljenja vremena na Internetu postavljanjem ranih ograničenja
pristupa nekim stranicama
(npr.~\emph{LeechBlock}\footnote{\url{https://addons.mozilla.org/en-US/firefox/addon/4476}}
dodatak za \emph{FireFox}\footnote{\url{http://www.mozilla.com/hr/}} web
preglednik) Ako ništa drugo ne djeluje, preporuča se isključivanje internetske
veze za vrijeme obavljanja posla.


\section{Zaključak}

Prokrastinacija je problem moderne informacijske svakodnevice u kojoj 
je pojedinac pretrpan velikim količinama lako dostupnih  informacija i 
izvora distrakcije i zabave. Označava, slikovito rečeno, ``problem započinjanja'' 
izvršavanja nekog zadatka, odnosno fenomen odgađanja ozbiljnog rada 
izvršavanjem kratkotrajnih, no ne nužno i nekonstruktivnih radnji. 

S obzirom na smjer u kojemu se razvijaju alati za obavljanje mnogih raznovrsnih 
zadataka, a koji uključuje značajan porast upotrebe računala i Interneta kao glavnih 
izvora prokrastinacije, ona postaje značajniji, ozbiljniji i rašireniji
problem. Prokrastinacija je aktualna tema brojnih socioloških i psiholoških 
istraživanja, a učinkovite metode borbe protiv ove problematike upravo zbog 
njezine sve veće zastupljenosti bitne su i uvelike utječu ne samo na radnu produktivnost 
pojednica, već i na samopercepciju osobe i njezino samopouzdanje. Više pozornosti 
trebalo bi se posvetiti ovoj temi posebice u okviru obrazovnih institucija.

\bibliography{literatura}
\bibliographystyle{plain}
\end{multicols}

\end{document}
